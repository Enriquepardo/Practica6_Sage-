
\documentclass{article}
\usepackage{listings}

\begin{document}

\begin{lstlisting}[language=Python]
import random
numeros = []
for i in range(1000):
  intervalo = random.randint(-100, 100)
  numeros.append(intervalo)

mean = sum(numeros) / len(numeros)
print(mean)
\end{lstlisting}

\begin{lstlisting}[language=Python]
alturas = [71, 73, 59, 62, 65, 61, 73, 61]
alturas.sort()
print('Media:' , mean(alturas))
print('Mediana:', median(alturas))
print('Moda:', mode(alturas))
\end{lstlisting}

\begin{lstlisting}[language=Python]
alturas = [71, 73, 59, 62, 65, 61, 73, 61]
print('Desviación típica:', std(alturas))
print('Varianza:',variance(alturas))
\end{lstlisting}

\begin{lstlisting}[language=Python]
alturas = [71, 73, 59, 62, 65, 61, 73, 61]
range = max(alturas) - min(alturas)
range
\end{lstlisting}

\end{document}
